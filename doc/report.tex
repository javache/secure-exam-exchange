\documentclass{article}

\usepackage{a4wide}
\usepackage{graphicx}
\usepackage{paralist}

\setlength{\parindent}{0em}
\setlength{\parskip}{1em}

\title{Project Information Security 2012}
\author{Pieter De Baets, Gilles Jacobs, Jasper Van der Jeugt, Toon Willems}

\begin{document}

\maketitle
\tableofcontents

\newpage

\section{Requirements and Architecture}

We chose to design system with a central component, i.e., a server, to which the
different clients can connect. One disadvantage of this approach is that this
server needs to be trusted to some extent---which we tried to limit as much as
possible, more on this later---by the users.

\subsection{Assumptions}

We need to assume that every user (students as well as professors) are
registered on this platform. These accounts can be created when a student
enrolls in the institure, or when a professor is employed.

We use an asymptotic encryption mechanism for many features, so we also need to
assume that each user has been a keypair assigned. These keypairs can be created
at the same time as the accounts, with some option to reset a keypair when it
has been compromised.

\subsection{Functionality}

We can see extract a number of use cases from the project specification:

\subsection{Expected attacks}

How will attackers try to compromise our system? Why will they fail?

\subsection{Limitations}

In what ways is our system not protected? What do we need to trust?

\section{Practical considerations}

Is it practically feasible to lock up a few dozen students in a small room for a
few days, without food or water?

\section{Concrete implementation}

Some implementation details, we should mention the actually used algorithms
here.

\end{document}
